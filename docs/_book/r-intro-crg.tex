\documentclass[]{book}
\usepackage{lmodern}
\usepackage{amssymb,amsmath}
\usepackage{ifxetex,ifluatex}
\usepackage{fixltx2e} % provides \textsubscript
\ifnum 0\ifxetex 1\fi\ifluatex 1\fi=0 % if pdftex
  \usepackage[T1]{fontenc}
  \usepackage[utf8]{inputenc}
\else % if luatex or xelatex
  \ifxetex
    \usepackage{mathspec}
  \else
    \usepackage{fontspec}
  \fi
  \defaultfontfeatures{Ligatures=TeX,Scale=MatchLowercase}
\fi
% use upquote if available, for straight quotes in verbatim environments
\IfFileExists{upquote.sty}{\usepackage{upquote}}{}
% use microtype if available
\IfFileExists{microtype.sty}{%
\usepackage{microtype}
\UseMicrotypeSet[protrusion]{basicmath} % disable protrusion for tt fonts
}{}
\usepackage{hyperref}
\hypersetup{unicode=true,
            pdftitle={Introduction to R},
            pdfauthor={Sarah Bonnin},
            pdfborder={0 0 0},
            breaklinks=true}
\urlstyle{same}  % don't use monospace font for urls
\usepackage{natbib}
\bibliographystyle{apalike}
\usepackage{longtable,booktabs}
\usepackage{graphicx,grffile}
\makeatletter
\def\maxwidth{\ifdim\Gin@nat@width>\linewidth\linewidth\else\Gin@nat@width\fi}
\def\maxheight{\ifdim\Gin@nat@height>\textheight\textheight\else\Gin@nat@height\fi}
\makeatother
% Scale images if necessary, so that they will not overflow the page
% margins by default, and it is still possible to overwrite the defaults
% using explicit options in \includegraphics[width, height, ...]{}
\setkeys{Gin}{width=\maxwidth,height=\maxheight,keepaspectratio}
\IfFileExists{parskip.sty}{%
\usepackage{parskip}
}{% else
\setlength{\parindent}{0pt}
\setlength{\parskip}{6pt plus 2pt minus 1pt}
}
\setlength{\emergencystretch}{3em}  % prevent overfull lines
\providecommand{\tightlist}{%
  \setlength{\itemsep}{0pt}\setlength{\parskip}{0pt}}
\setcounter{secnumdepth}{5}
% Redefines (sub)paragraphs to behave more like sections
\ifx\paragraph\undefined\else
\let\oldparagraph\paragraph
\renewcommand{\paragraph}[1]{\oldparagraph{#1}\mbox{}}
\fi
\ifx\subparagraph\undefined\else
\let\oldsubparagraph\subparagraph
\renewcommand{\subparagraph}[1]{\oldsubparagraph{#1}\mbox{}}
\fi

%%% Use protect on footnotes to avoid problems with footnotes in titles
\let\rmarkdownfootnote\footnote%
\def\footnote{\protect\rmarkdownfootnote}

%%% Change title format to be more compact
\usepackage{titling}

% Create subtitle command for use in maketitle
\providecommand{\subtitle}[1]{
  \posttitle{
    \begin{center}\large#1\end{center}
    }
}

\setlength{\droptitle}{-2em}

  \title{Introduction to R}
    \pretitle{\vspace{\droptitle}\centering\huge}
  \posttitle{\par}
    \author{Sarah Bonnin}
    \preauthor{\centering\large\emph}
  \postauthor{\par}
      \predate{\centering\large\emph}
  \postdate{\par}
    \date{2019-08-14}

\usepackage{booktabs}
\usepackage{amsthm}
\makeatletter
\def\thm@space@setup{%
  \thm@preskip=8pt plus 2pt minus 4pt
  \thm@postskip=\thm@preskip
}
\makeatother

\begin{document}
\maketitle

{
\setcounter{tocdepth}{1}
\tableofcontents
}
\chapter{Before we start}\label{before-we-start}

Dates, time \& location

Dates:

\begin{itemize}
\tightlist
\item
  Module 1: Monday 18th \& Tuesday 19th February, 2019
\item
  Module 2: Monday 25th \& Tuesday 26th February, 2019
\item
  Module 3: Monday 4th \& Tuesday 5th March, 2019
\item
  Module 4: Monday 11th \& Tuesday 12th March, 2019
\end{itemize}

Time:

\begin{itemize}
\tightlist
\item
  10:00-13:30 
\end{itemize}

Location:

\begin{itemize}
\tightlist
\item
  CRG Training center
\end{itemize}

Instructors

\href{mailto:julia.ponomarenko@crg.eu}{Julia Ponomarenko} (Module 4)
\href{mailto:sarah.bonnin@crg.eu}{Sarah Bonnin} (Module 1, 2, 3) from
the CRG \href{https://biocore.crg.eu/}{Bioinformatics core facility}
(office 460, 4th floor hotel side)

Learning objectives

\chapter{What is R ?}\label{what-is-r}

\begin{itemize}
\item
  Programming language and environment for \textbf{data manipulation},
  \textbf{statistical computing}, and \textbf{graphical display}.
\item
  Implementation of the S programming language
\item
  Created at the University of Auckland, New Zealand:

  \begin{itemize}
  \tightlist
  \item
    Initial version released in 1995
  \item
    Stable version released in 2000
  \end{itemize}
\item
  \textbf{Free and open source !}

  \begin{itemize}
  \tightlist
  \item
    \url{https://www.r-project.org/}
  \end{itemize}
\item
  Interactive, flexible
\item
  Very active community of developers and users!

  \begin{itemize}
  \tightlist
  \item
    Many resources and forums available
  \end{itemize}
\item
  Access through a command-line interpreter:
\end{itemize}

\chapter{R Studio}\label{r-studio}

\begin{itemize}
\item
  Free and open source IDE (Integrated Development Environment) for R
\item
  Available for Windows, Mac OS and LINUX
\end{itemize}

RStudio access

\begin{itemize}
\item
  \href{https://www.rstudio.com/products/rstudio/download}{RStudio
  Desktop installation}
\item
  \href{http://rstudio.linux.crg.es/}{RStudio access from the CRG
  server}

  \begin{itemize}
  \tightlist
  \item
    Access with CRG credentials
  \item
    For those who don't have access to the CRG server, use the guest
    accounts.
  \end{itemize}
\end{itemize}

RStudio interface

\begin{itemize}
\tightlist
\item
  4 panels:

  \begin{itemize}
  \tightlist
  \item
    top-left: scripts and files
  \item
    bottom-left: R terminal
  \item
    top-right: objects, history and environment
  \item
    bottom-right: tree of folders, graph window, packages, help window,
    viewer
  \end{itemize}
\end{itemize}

Setting up the folder structure for the course

Rcourse \textbar{}-Module1 \textbar{}-Module2 \textbar{}-Module3
\textbar{}-Module4

\chapter{Paths and directories}\label{paths-and-directories}

\chapter{R basics}\label{r-basics}

\chapter{Functions basics}\label{functions-basics}

\chapter{R scripts}\label{r-scripts}

\chapter{Exercise 1}\label{exercise-1}

\chapter{Data types}\label{data-types}

\bibliography{book.bib,packages.bib}


\end{document}
